% Options for packages loaded elsewhere
\PassOptionsToPackage{unicode}{hyperref}
\PassOptionsToPackage{hyphens}{url}
%
\documentclass[
]{article}
\usepackage{amsmath,amssymb}
\usepackage{iftex}
\ifPDFTeX
  \usepackage[T1]{fontenc}
  \usepackage[utf8]{inputenc}
  \usepackage{textcomp} % provide euro and other symbols
\else % if luatex or xetex
  \usepackage{unicode-math} % this also loads fontspec
  \defaultfontfeatures{Scale=MatchLowercase}
  \defaultfontfeatures[\rmfamily]{Ligatures=TeX,Scale=1}
\fi
\usepackage{lmodern}
\ifPDFTeX\else
  % xetex/luatex font selection
\fi
% Use upquote if available, for straight quotes in verbatim environments
\IfFileExists{upquote.sty}{\usepackage{upquote}}{}
\IfFileExists{microtype.sty}{% use microtype if available
  \usepackage[]{microtype}
  \UseMicrotypeSet[protrusion]{basicmath} % disable protrusion for tt fonts
}{}
\makeatletter
\@ifundefined{KOMAClassName}{% if non-KOMA class
  \IfFileExists{parskip.sty}{%
    \usepackage{parskip}
  }{% else
    \setlength{\parindent}{0pt}
    \setlength{\parskip}{6pt plus 2pt minus 1pt}}
}{% if KOMA class
  \KOMAoptions{parskip=half}}
\makeatother
\usepackage{xcolor}
\usepackage[margin=1in]{geometry}
\usepackage{color}
\usepackage{fancyvrb}
\newcommand{\VerbBar}{|}
\newcommand{\VERB}{\Verb[commandchars=\\\{\}]}
\DefineVerbatimEnvironment{Highlighting}{Verbatim}{commandchars=\\\{\}}
% Add ',fontsize=\small' for more characters per line
\usepackage{framed}
\definecolor{shadecolor}{RGB}{248,248,248}
\newenvironment{Shaded}{\begin{snugshade}}{\end{snugshade}}
\newcommand{\AlertTok}[1]{\textcolor[rgb]{0.94,0.16,0.16}{#1}}
\newcommand{\AnnotationTok}[1]{\textcolor[rgb]{0.56,0.35,0.01}{\textbf{\textit{#1}}}}
\newcommand{\AttributeTok}[1]{\textcolor[rgb]{0.13,0.29,0.53}{#1}}
\newcommand{\BaseNTok}[1]{\textcolor[rgb]{0.00,0.00,0.81}{#1}}
\newcommand{\BuiltInTok}[1]{#1}
\newcommand{\CharTok}[1]{\textcolor[rgb]{0.31,0.60,0.02}{#1}}
\newcommand{\CommentTok}[1]{\textcolor[rgb]{0.56,0.35,0.01}{\textit{#1}}}
\newcommand{\CommentVarTok}[1]{\textcolor[rgb]{0.56,0.35,0.01}{\textbf{\textit{#1}}}}
\newcommand{\ConstantTok}[1]{\textcolor[rgb]{0.56,0.35,0.01}{#1}}
\newcommand{\ControlFlowTok}[1]{\textcolor[rgb]{0.13,0.29,0.53}{\textbf{#1}}}
\newcommand{\DataTypeTok}[1]{\textcolor[rgb]{0.13,0.29,0.53}{#1}}
\newcommand{\DecValTok}[1]{\textcolor[rgb]{0.00,0.00,0.81}{#1}}
\newcommand{\DocumentationTok}[1]{\textcolor[rgb]{0.56,0.35,0.01}{\textbf{\textit{#1}}}}
\newcommand{\ErrorTok}[1]{\textcolor[rgb]{0.64,0.00,0.00}{\textbf{#1}}}
\newcommand{\ExtensionTok}[1]{#1}
\newcommand{\FloatTok}[1]{\textcolor[rgb]{0.00,0.00,0.81}{#1}}
\newcommand{\FunctionTok}[1]{\textcolor[rgb]{0.13,0.29,0.53}{\textbf{#1}}}
\newcommand{\ImportTok}[1]{#1}
\newcommand{\InformationTok}[1]{\textcolor[rgb]{0.56,0.35,0.01}{\textbf{\textit{#1}}}}
\newcommand{\KeywordTok}[1]{\textcolor[rgb]{0.13,0.29,0.53}{\textbf{#1}}}
\newcommand{\NormalTok}[1]{#1}
\newcommand{\OperatorTok}[1]{\textcolor[rgb]{0.81,0.36,0.00}{\textbf{#1}}}
\newcommand{\OtherTok}[1]{\textcolor[rgb]{0.56,0.35,0.01}{#1}}
\newcommand{\PreprocessorTok}[1]{\textcolor[rgb]{0.56,0.35,0.01}{\textit{#1}}}
\newcommand{\RegionMarkerTok}[1]{#1}
\newcommand{\SpecialCharTok}[1]{\textcolor[rgb]{0.81,0.36,0.00}{\textbf{#1}}}
\newcommand{\SpecialStringTok}[1]{\textcolor[rgb]{0.31,0.60,0.02}{#1}}
\newcommand{\StringTok}[1]{\textcolor[rgb]{0.31,0.60,0.02}{#1}}
\newcommand{\VariableTok}[1]{\textcolor[rgb]{0.00,0.00,0.00}{#1}}
\newcommand{\VerbatimStringTok}[1]{\textcolor[rgb]{0.31,0.60,0.02}{#1}}
\newcommand{\WarningTok}[1]{\textcolor[rgb]{0.56,0.35,0.01}{\textbf{\textit{#1}}}}
\usepackage{graphicx}
\makeatletter
\def\maxwidth{\ifdim\Gin@nat@width>\linewidth\linewidth\else\Gin@nat@width\fi}
\def\maxheight{\ifdim\Gin@nat@height>\textheight\textheight\else\Gin@nat@height\fi}
\makeatother
% Scale images if necessary, so that they will not overflow the page
% margins by default, and it is still possible to overwrite the defaults
% using explicit options in \includegraphics[width, height, ...]{}
\setkeys{Gin}{width=\maxwidth,height=\maxheight,keepaspectratio}
% Set default figure placement to htbp
\makeatletter
\def\fps@figure{htbp}
\makeatother
\setlength{\emergencystretch}{3em} % prevent overfull lines
\providecommand{\tightlist}{%
  \setlength{\itemsep}{0pt}\setlength{\parskip}{0pt}}
\setcounter{secnumdepth}{-\maxdimen} % remove section numbering
\usepackage{booktabs}
\usepackage{longtable}
\usepackage{array}
\usepackage{multirow}
\usepackage{wrapfig}
\usepackage{float}
\usepackage{colortbl}
\usepackage{pdflscape}
\usepackage{tabu}
\usepackage{threeparttable}
\usepackage{threeparttablex}
\usepackage[normalem]{ulem}
\usepackage{makecell}
\usepackage{xcolor}
\ifLuaTeX
  \usepackage{selnolig}  % disable illegal ligatures
\fi
\usepackage{bookmark}
\IfFileExists{xurl.sty}{\usepackage{xurl}}{} % add URL line breaks if available
\urlstyle{same}
\hypersetup{
  pdftitle={Exercici d'events recurrents en R (part II)},
  pdfauthor={Jose Calatayud Mateu},
  hidelinks,
  pdfcreator={LaTeX via pandoc}}

\title{Exercici d'events recurrents en R (part II)}
\author{Jose Calatayud Mateu}
\date{}

\begin{document}
\maketitle

\begin{Shaded}
\begin{Highlighting}[]
\DocumentationTok{\#\# Llibreria}
\FunctionTok{library}\NormalTok{(devtools)}
\FunctionTok{library}\NormalTok{(tidyverse)}
\NormalTok{devtools}\SpecialCharTok{::}\FunctionTok{install\_github}\NormalTok{(}\StringTok{"isglobal{-}brge/survrec"}\NormalTok{, }
                         \AttributeTok{build =} \ConstantTok{FALSE}\NormalTok{)}

\FunctionTok{library}\NormalTok{(survrec)}
\FunctionTok{library}\NormalTok{(kableExtra)}
\end{Highlighting}
\end{Shaded}

Procedim a descarregar manualment l'arxius assosciats als paquet
``gcmrec'' que està dedicat a funcions de modelatge de models generals
per events recurrents i el data sobre el qual treballarem ``lymphoma'':

\begin{Shaded}
\begin{Highlighting}[]
\FunctionTok{source}\NormalTok{(}\AttributeTok{file=}\StringTok{"lymphoma.R"}\NormalTok{)}
\FunctionTok{source}\NormalTok{(}\AttributeTok{file=}\StringTok{"survrec.R"}\NormalTok{)}
\FunctionTok{source}\NormalTok{(}\AttributeTok{file=}\StringTok{"gcmrec.R"}\NormalTok{)}
\end{Highlighting}
\end{Shaded}

El data ``lymphoma'' contenen els temps de recaiguda del càncer després
del primer tractamenten pacients diagnosticats amb lymphoma de grau
baix.

\begin{Shaded}
\begin{Highlighting}[]
\FunctionTok{head}\NormalTok{(lymphoma)}
\end{Highlighting}
\end{Shaded}

\begin{verbatim}
##   id      time event enum delay age sex distrib effage
## 1  6  3.900826     0    1    17  79   1       1     CR
## 2  7 63.173554     0    1    33  25   1       1     CR
## 3  8 41.289256     0    1    26  37   1       2     CR
## 4 11 29.421488     1    1    31  43   2       2     CR
## 5 11 20.826446     1    2    31  43   2       2     CR
## 6 11 17.950413     0    3    31  43   2       2     CR
\end{verbatim}

\emph{NOTA}: la variable \emph{time} conté els temps entre
esdeveniments, \emph{event} és la variable de censura que val 1 per a
recaigudes de càncer i 0 per al darrer moemnt de seguiment (indicant que
l'esdeveniment no s'ha observat), i la variable \emph{id} identifica
cada pacient.

\subsection{Exercici 1:}\label{exercici-1}

\begin{itemize}
\tightlist
\item
  \emph{Estime los modelos AG, PWP-Gap time, PWP-Total time y WLW para
  determinar si existen diferencias en el tiempo de recaída en función
  del número de lesiones al momento del diagnóstico (variable distrib).
  NOTA: Deberá crear los marcos de datos adecuados para los modelos
  PWT-Gap time y WLW.}
\end{itemize}

Utilizando los datos de lyphoma (package \texttt{gcmrec}), determinar si
hay diferències en el temps de recaiguida segons el número de lesions al
diagnóstic (variable \texttt{distrib}). Per dur-ho a terme, es tenen que
estimar els models AG, PWP-Gap time, PWP-Total time y WLW.

El conjunt de dades contenen les següents variables:

\begin{itemize}
\tightlist
\item
  \texttt{id}: identificador del pacient
\item
  \texttt{time}: temps entre recaiguda
\item
  \texttt{event}: variable de censura (1 = recaiguda, 0 = últim
  seguiment sense observar res).
\item
  \texttt{distrib}: nombre de lesions al diagnostic (0 = única, 1 =
  localitzada, 2 = més d'un lloc nodal, 3 = generalitzada).
\item
  \texttt{sex}: sexo del pacient.
\item
  \texttt{treat}: resposta al tratament.
\item
  \texttt{age}: edat del pacient.
\end{itemize}

\paragraph{\texorpdfstring{\textbf{Preparar los datos para los
diferentes
modelos}}{Preparar los datos para los diferentes modelos}}\label{preparar-los-datos-para-los-diferentes-modelos}

\subparagraph{1. Model AG (Andersen-Gill)}\label{model-ag-andersen-gill}

El model AG és una extensió del model de Cox proporcional de risc per a
dades d'events recurrents. El seu objectiu és modelar el risc
d'ocurrència d'un event recurrent, considerant que un subjecte pot
experimentar múltiples events durant el periode de seguiment. A
continuació, es detalla la seua estructura i suposits

El model AG asumeix que els events recurrents d'un subjecte segueixen un
procès de recompte amb increments independents. El risc per al k-èssim
event del subjecte i es modela com:

\[
\lambda_k(t;Z_{ik}) = \lambda_0(t) e^{\beta'Z_{it}(t)}
\] d'on:

\begin{itemize}
\tightlist
\item
  \(\lambda_0(t)\) és la funció de risc basal comú a tots els individus
\item
  \(Z_{ik}(t)\) és un vector de covariables per al individu i en el
  k-èssim event.
\item
  \(\beta\) és el vector de paràmetres a estimar
\end{itemize}

El model AG presenta els següent suposits en la seua construcció:

\begin{itemize}
\item
  \emph{Independència dels increments}: El model AG assumeix que els
  temps entre esdeveniments consecutius son independents. Açò significa
  que, el risc d'un nou esdeveniment no depen d'esdeveniments previs,
  almenys que s'incloguen covariables que capturen eixa dependència (per
  example, el número d'esdeveniments previs com covariable del tipus
  temps)
\item
  \emph{Risc basal comú}: Es usposa que tots els individus comparteixen
  la mateixa funció de risc basal \(\lambda_0(t)\). Açò pot ser un
  limitació si el risc basalvaria entre esdeveniments
\item
  \emph{Efecte proporcional de les covariables}: Les covariable tenen un
  efecte multiplicatiu constant sobre el risc, independenment del número
  d'esdevenimients previs
\end{itemize}

Per poder dur a terme el model AG, les dades tenen que estar en format
de recompte, és a dir, que cada fila representa un interval de temps
entre esdeveniments consecutius i esto ho podem fer amb la funció
\texttt{getCountingProcess} de la següent manera:

\begin{Shaded}
\begin{Highlighting}[]
\CommentTok{\# getCountingProcess: convertir les dades en format de recompte}
\FunctionTok{library}\NormalTok{(tidyverse)}
\NormalTok{getCountingProcess }\OtherTok{\textless{}{-}} \ControlFlowTok{function}\NormalTok{(x, colID, colTime) \{}
\NormalTok{  id }\OtherTok{\textless{}{-}}\NormalTok{ x[, colID]}
\NormalTok{  ids }\OtherTok{\textless{}{-}} \FunctionTok{unique}\NormalTok{(id)}
\NormalTok{  tt }\OtherTok{\textless{}{-}}\NormalTok{ x[,colTime]}
\NormalTok{  out }\OtherTok{\textless{}{-}} \ConstantTok{NULL}
  \ControlFlowTok{for}\NormalTok{ (i }\ControlFlowTok{in} \DecValTok{1}\SpecialCharTok{:}\FunctionTok{length}\NormalTok{(ids))\{}
\NormalTok{    tt.i }\OtherTok{\textless{}{-}}\NormalTok{ tt[id}\SpecialCharTok{\%in\%}\NormalTok{ids[i]]}
\NormalTok{    start }\OtherTok{\textless{}{-}} \FunctionTok{c}\NormalTok{(}\DecValTok{0}\NormalTok{, tt.i[}\SpecialCharTok{{-}}\FunctionTok{length}\NormalTok{(tt.i)])}
\NormalTok{    out.i }\OtherTok{\textless{}{-}} \FunctionTok{cbind}\NormalTok{(start, tt.i)}
\NormalTok{    out }\OtherTok{\textless{}{-}} \FunctionTok{rbind}\NormalTok{(out, out.i)}
\NormalTok{  \}}
\NormalTok{  ans }\OtherTok{\textless{}{-}} \FunctionTok{data.frame}\NormalTok{(x[,colID,}\AttributeTok{drop=}\ConstantTok{TRUE}\NormalTok{],}
\NormalTok{                    x[,colTime],}\AttributeTok{start=}\NormalTok{out[,}\DecValTok{1}\NormalTok{],}\AttributeTok{stop=}\NormalTok{out[,}\DecValTok{2}\NormalTok{],}
\NormalTok{                    x }\SpecialCharTok{\%\textgreater{}\%}\NormalTok{ dplyr}\SpecialCharTok{::}\FunctionTok{select}\NormalTok{(}\SpecialCharTok{!}\FunctionTok{c}\NormalTok{(colID, colTime)))}
  
  \FunctionTok{names}\NormalTok{(ans)[}\DecValTok{1}\SpecialCharTok{:}\DecValTok{2}\NormalTok{] }\OtherTok{\textless{}{-}} \FunctionTok{names}\NormalTok{(x[,}\FunctionTok{c}\NormalTok{(colID, colTime)])}
  \FunctionTok{return}\NormalTok{(ans)}
\NormalTok{\}}
\end{Highlighting}
\end{Shaded}

Abans de res, les dades deuen ordenar-se per \texttt{id} i \texttt{time}
per garantir un anàlisis adequat d'events recurrents:

\begin{Shaded}
\begin{Highlighting}[]
\NormalTok{lymphoma }\OtherTok{\textless{}{-}}\NormalTok{ lymphoma[}\FunctionTok{order}\NormalTok{(lymphoma}\SpecialCharTok{$}\NormalTok{id, lymphoma}\SpecialCharTok{$}\NormalTok{time), ]}
\end{Highlighting}
\end{Shaded}

Apliquem la funció \texttt{getCountingProcess} per obtendre el data en
el format correcte per les meues dades:

\begin{Shaded}
\begin{Highlighting}[]
\CommentTok{\# Aplicar a les nostres dades}
\NormalTok{lymphoma\_ag }\OtherTok{\textless{}{-}} \FunctionTok{getCountingProcess}\NormalTok{(lymphoma, }\AttributeTok{colID =} \StringTok{"id"}\NormalTok{, }\AttributeTok{colTime =} \StringTok{"time"}\NormalTok{)}
\end{Highlighting}
\end{Shaded}

\begin{verbatim}
## Warning: Using an external vector in selections was deprecated in tidyselect 1.1.0.
## i Please use `all_of()` or `any_of()` instead.
##   # Was:
##   data %>% select(colID)
## 
##   # Now:
##   data %>% select(all_of(colID))
## 
## See <https://tidyselect.r-lib.org/reference/faq-external-vector.html>.
## This warning is displayed once every 8 hours.
## Call `lifecycle::last_lifecycle_warnings()` to see where this warning was
## generated.
\end{verbatim}

\begin{verbatim}
## Warning: Using an external vector in selections was deprecated in tidyselect 1.1.0.
## i Please use `all_of()` or `any_of()` instead.
##   # Was:
##   data %>% select(colTime)
## 
##   # Now:
##   data %>% select(all_of(colTime))
## 
## See <https://tidyselect.r-lib.org/reference/faq-external-vector.html>.
## This warning is displayed once every 8 hours.
## Call `lifecycle::last_lifecycle_warnings()` to see where this warning was
## generated.
\end{verbatim}

\begin{Shaded}
\begin{Highlighting}[]
\FunctionTok{head}\NormalTok{(lymphoma\_ag)}
\end{Highlighting}
\end{Shaded}

\begin{verbatim}
##   id      time    start      stop event enum delay age sex distrib effage
## 1  6  3.900826  0.00000  3.900826     0    1    17  79   1       1     CR
## 2  7 63.173554  0.00000 63.173554     0    1    33  25   1       1     CR
## 3  8 41.289256  0.00000 41.289256     0    1    26  37   1       2     CR
## 6 11 17.950413  0.00000 17.950413     0    3    31  43   2       2     CR
## 5 11 20.826446 17.95041 20.826446     1    2    31  43   2       2     CR
## 4 11 29.421488 20.82645 29.421488     1    1    31  43   2       2     CR
\end{verbatim}

El model AG per avaluar si el número de lesions al diagnóstic s'associa
amb els temps de recaiguda del càncer després del primer tractamenten
pacients diagnosticats amb lymphoma de grau baix s'ajusta ampliant el
model Cox implementat al \texttt{survival} package:

\begin{Shaded}
\begin{Highlighting}[]
\FunctionTok{library}\NormalTok{(survival)}
\end{Highlighting}
\end{Shaded}

\begin{verbatim}
## 
## Attaching package: 'survival'
\end{verbatim}

\begin{verbatim}
## The following object is masked from 'package:boot':
## 
##     aml
\end{verbatim}

\begin{Shaded}
\begin{Highlighting}[]
\CommentTok{\# Ajustar el modelo AG}
\NormalTok{ag\_model }\OtherTok{\textless{}{-}} \FunctionTok{coxph}\NormalTok{(}\FunctionTok{Surv}\NormalTok{(start, stop, event) }\SpecialCharTok{\textasciitilde{}} \FunctionTok{as.factor}\NormalTok{(distrib) }\SpecialCharTok{+} \FunctionTok{cluster}\NormalTok{(id), }\AttributeTok{data =}\NormalTok{ lymphoma\_ag)}
\FunctionTok{summary}\NormalTok{(ag\_model)}
\end{Highlighting}
\end{Shaded}

\begin{verbatim}
## Call:
## coxph(formula = Surv(start, stop, event) ~ as.factor(distrib), 
##     data = lymphoma_ag, cluster = id)
## 
##   n= 110, number of events= 49 
##    (2 observations deleted due to missingness)
## 
##                       coef exp(coef) se(coef) robust se     z Pr(>|z|)   
## as.factor(distrib)1 1.0304    2.8021   0.4987    0.4801 2.146  0.03187 * 
## as.factor(distrib)2 1.4776    4.3826   0.5036    0.4775 3.094  0.00197 **
## as.factor(distrib)3 1.2377    3.4478   0.7400    0.6367 1.944  0.05190 . 
## ---
## Signif. codes:  0 '***' 0.001 '**' 0.01 '*' 0.05 '.' 0.1 ' ' 1
## 
##                     exp(coef) exp(-coef) lower .95 upper .95
## as.factor(distrib)1     2.802     0.3569    1.0935      7.18
## as.factor(distrib)2     4.383     0.2282    1.7189     11.17
## as.factor(distrib)3     3.448     0.2900    0.9899     12.01
## 
## Concordance= 0.632  (se = 0.042 )
## Likelihood ratio test= 10.98  on 3 df,   p=0.01
## Wald test            = 9.76  on 3 df,   p=0.02
## Score (logrank) test = 10  on 3 df,   p=0.02,   Robust = 7.87  p=0.05
## 
##   (Note: the likelihood ratio and score tests assume independence of
##      observations within a cluster, the Wald and robust score tests do not).
\end{verbatim}

Anem a interpretar els resultats del model AG:

\begin{itemize}
\item
  Coeficients: els coeficients estimats \(\beta\) indiquen que l'efecte
  de cada categoria de \texttt{distrib} sobre el risc de recaiguda.
\item
  Exp(coef): És la interpretació en termes de risc relatiu respecte la
  categoria basal. En el nostre cas, la categoria basal és quan la
  lessió durant el diagnóstic és única.
\end{itemize}

Amb els resultat del model, es pot veure que totes les categories de
lessions durant el diagnostic son covariables significatives excepte la
generalitzada.

A l'hora d'escollir un model per al temps de recaiguda del cancer, és
important tenir en compte el procés de la malaltia. Si després de patir
la primera recaiguda, el riscde la següent pot augmentar, això suggereix
un model que incorpori una covariable dependent del temps per al nombre
de recaigudes o utilitzar un model que contingui estrats separats per a
cada esdeveniment com el model PWP. Tanmateix, si el risc de recaure es
manté constant independentment del nombre de recaigudes anteriors, el
model AG seria adequat. Per investigar la dependència dels
esdeveniments, hem de crear una variable dependent del temps (enum) que
codifiqui el nombre de recaigudes. Aquesta variable es pot crear
mitjançant aquesta funció:

\begin{Shaded}
\begin{Highlighting}[]
\CommentTok{\# getEnum: indica el número de recaigudes}
\NormalTok{getEnum }\OtherTok{\textless{}{-}} \ControlFlowTok{function}\NormalTok{(x, colID) \{}
\NormalTok{  id }\OtherTok{\textless{}{-}}\NormalTok{ x[, colID]}
\NormalTok{  ids }\OtherTok{\textless{}{-}} \FunctionTok{unique}\NormalTok{(id)}
\NormalTok{  out }\OtherTok{\textless{}{-}} \ConstantTok{NULL}
  \ControlFlowTok{for}\NormalTok{ (i }\ControlFlowTok{in} \DecValTok{1}\SpecialCharTok{:}\FunctionTok{length}\NormalTok{(ids)) \{}
\NormalTok{    tt.i }\OtherTok{\textless{}{-}} \FunctionTok{sum}\NormalTok{(id }\SpecialCharTok{\%in\%}\NormalTok{ ids[i])}
\NormalTok{    out.i }\OtherTok{\textless{}{-}} \DecValTok{1}\SpecialCharTok{:}\NormalTok{tt.i}
\NormalTok{    out }\OtherTok{\textless{}{-}} \FunctionTok{c}\NormalTok{(out, out.i)}
\NormalTok{  \}}
\NormalTok{  out}
\NormalTok{\}}
\end{Highlighting}
\end{Shaded}

Aleshores, incorporem aquesta variable dependent del temps al model de
recaigudes per analitzar la dependència entre els temps de recurrència.

\begin{Shaded}
\begin{Highlighting}[]
\NormalTok{lymphoma\_ag}\SpecialCharTok{$}\NormalTok{enum }\OtherTok{\textless{}{-}} \FunctionTok{getEnum}\NormalTok{(lymphoma\_ag, }\AttributeTok{colID=}\DecValTok{1}\NormalTok{)}
\FunctionTok{head}\NormalTok{(lymphoma\_ag)}
\end{Highlighting}
\end{Shaded}

\begin{verbatim}
##   id      time    start      stop event enum delay age sex distrib effage
## 1  6  3.900826  0.00000  3.900826     0    1    17  79   1       1     CR
## 2  7 63.173554  0.00000 63.173554     0    1    33  25   1       1     CR
## 3  8 41.289256  0.00000 41.289256     0    1    26  37   1       2     CR
## 6 11 17.950413  0.00000 17.950413     0    1    31  43   2       2     CR
## 5 11 20.826446 17.95041 20.826446     1    2    31  43   2       2     CR
## 4 11 29.421488 20.82645 29.421488     1    3    31  43   2       2     CR
\end{verbatim}

Fent el nou model incorporant la covariable \texttt{enum}:

\begin{Shaded}
\begin{Highlighting}[]
\NormalTok{ag.fit }\OtherTok{\textless{}{-}} \FunctionTok{coxph}\NormalTok{(}\FunctionTok{Surv}\NormalTok{(start, stop, event) }\SpecialCharTok{\textasciitilde{}} \FunctionTok{as.factor}\NormalTok{(distrib)}\SpecialCharTok{+}
                  \FunctionTok{cluster}\NormalTok{(id), }\AttributeTok{data=}\FunctionTok{subset}\NormalTok{(lymphoma\_ag, enum}\SpecialCharTok{\textless{}}\DecValTok{4}\NormalTok{))}
\end{Highlighting}
\end{Shaded}

\begin{verbatim}
## Warning in Surv(start, stop, event): Stop time must be > start time, NA created
\end{verbatim}

\begin{Shaded}
\begin{Highlighting}[]
\NormalTok{ag.fit}
\end{Highlighting}
\end{Shaded}

\begin{verbatim}
## Call:
## coxph(formula = Surv(start, stop, event) ~ as.factor(distrib), 
##     data = subset(lymphoma_ag, enum < 4), cluster = id)
## 
##                       coef exp(coef) se(coef) robust se     z       p
## as.factor(distrib)1 0.9472    2.5784   0.5099    0.5002 1.894 0.05827
## as.factor(distrib)2 1.3789    3.9704   0.5105    0.4759 2.897 0.00376
## as.factor(distrib)3 1.4745    4.3689   0.7322    0.7090 2.080 0.03755
## 
## Likelihood ratio test=9.68  on 3 df, p=0.02152
## n= 101, number of events= 43 
##    (2 observations deleted due to missingness)
\end{verbatim}

\begin{Shaded}
\begin{Highlighting}[]
\NormalTok{ag.fit.dep }\OtherTok{\textless{}{-}} \FunctionTok{coxph}\NormalTok{(}\FunctionTok{Surv}\NormalTok{(start, stop, event) }\SpecialCharTok{\textasciitilde{}} \FunctionTok{as.factor}\NormalTok{(distrib) }\SpecialCharTok{+} \FunctionTok{cluster}\NormalTok{(id) }\SpecialCharTok{+}\NormalTok{ enum, }\AttributeTok{data=}\FunctionTok{subset}\NormalTok{(lymphoma\_ag, enum }\SpecialCharTok{\textless{}} \DecValTok{4}\NormalTok{))}
\end{Highlighting}
\end{Shaded}

\begin{verbatim}
## Warning in Surv(start, stop, event): Stop time must be > start time, NA created
\end{verbatim}

\begin{Shaded}
\begin{Highlighting}[]
\NormalTok{ag.fit.dep}
\end{Highlighting}
\end{Shaded}

\begin{verbatim}
## Call:
## coxph(formula = Surv(start, stop, event) ~ as.factor(distrib) + 
##     enum, data = subset(lymphoma_ag, enum < 4), cluster = id)
## 
##                       coef exp(coef) se(coef) robust se     z       p
## as.factor(distrib)1 0.5944    1.8120   0.5260    0.4383 1.356   0.175
## as.factor(distrib)2 0.7600    2.1383   0.5450    0.4668 1.628   0.103
## as.factor(distrib)3 0.7721    2.1644   0.7597    0.5516 1.400   0.162
## enum                0.8676    2.3812   0.2053    0.2209 3.927 8.6e-05
## 
## Likelihood ratio test=27.26  on 4 df, p=1.76e-05
## n= 101, number of events= 43 
##    (2 observations deleted due to missingness)
\end{verbatim}

Incloure la variable dependent del temps que té en compte el nombre de
recaigudes previs (enum) redueix el valor de l'estadística \(−2logL\)
(per exemple, la desviació) que mesura la bondat d'ajust d'un model.

\begin{Shaded}
\begin{Highlighting}[]
\NormalTok{deviance}\FloatTok{.1} \OtherTok{\textless{}{-}} \SpecialCharTok{{-}}\DecValTok{2}\SpecialCharTok{*}\FunctionTok{summary}\NormalTok{(ag.fit)}\SpecialCharTok{$}\NormalTok{loglik[}\DecValTok{2}\NormalTok{] }
\NormalTok{deviance}\FloatTok{.2} \OtherTok{\textless{}{-}} \SpecialCharTok{{-}}\DecValTok{2}\SpecialCharTok{*}\FunctionTok{summary}\NormalTok{(ag.fit.dep)}\SpecialCharTok{$}\NormalTok{loglik[}\DecValTok{2}\NormalTok{] }
\FunctionTok{pchisq}\NormalTok{(deviance}\FloatTok{.1} \SpecialCharTok{{-}}\NormalTok{ deviance}\FloatTok{.2}\NormalTok{, }\AttributeTok{df=}\DecValTok{1}\NormalTok{, }\AttributeTok{lower=}\ConstantTok{FALSE}\NormalTok{)}
\end{Highlighting}
\end{Shaded}

\begin{verbatim}
## [1] 2.747985e-05
\end{verbatim}

\begin{Shaded}
\begin{Highlighting}[]
\FunctionTok{qchisq}\NormalTok{(}\AttributeTok{p=}\FloatTok{0.95}\NormalTok{, }\AttributeTok{df=}\DecValTok{1}\NormalTok{, }\AttributeTok{lower=}\ConstantTok{FALSE}\NormalTok{)}
\end{Highlighting}
\end{Shaded}

\begin{verbatim}
## [1] 0.00393214
\end{verbatim}

que es estadísticamente significatiu (\(95\%\)).

Per tant, el risc de recaiguda de cancer augmenta un \(70\%\)
\(exp(0.5337)=1.70\) a mesura que augmenta el nombre recaigudes previes.
El model AG assumeix una funció de risc de referència comuna per a totes
les recaigudes. Per tant, podem pensar que utilitzar un model
estratificat (model PWP) per nombre de recaigudes podria ser més
apropiat. Això estima el model de temps total PWP, ja que les variables
d'aturada estan anotades en el temps de calendari.

\subparagraph{2. Model PWP-Total time}\label{model-pwp-total-time}

En l'apartat anterior amb el model AG assumim una funció de risc de
referència per a totes les recaigudes. Si volem pensar en utilitzar un
model estratificant pel nombre de recaigudes aleshores el PWP model és
més apropiat, és a dir, estimant el model de PWP total time fins que
l'stop variables.

D'aquest forma, el nou model consistirà en la incorporació de la
variable \texttt{enum} com a estrat dintre del model AG per obtindre el
model PWP-Total time:

\begin{Shaded}
\begin{Highlighting}[]
\NormalTok{pwp.fit.total }\OtherTok{\textless{}{-}} \FunctionTok{coxph}\NormalTok{(}\FunctionTok{Surv}\NormalTok{(start, stop, event) }\SpecialCharTok{\textasciitilde{}} \FunctionTok{as.factor}\NormalTok{(distrib) }\SpecialCharTok{+} \FunctionTok{cluster}\NormalTok{(id) }\SpecialCharTok{+} \FunctionTok{strata}\NormalTok{(enum), }\AttributeTok{data=}\FunctionTok{subset}\NormalTok{(lymphoma\_ag, enum }\SpecialCharTok{\textless{}} \DecValTok{4}\NormalTok{))}
\end{Highlighting}
\end{Shaded}

\begin{verbatim}
## Warning in Surv(start, stop, event): Stop time must be > start time, NA created
\end{verbatim}

\begin{Shaded}
\begin{Highlighting}[]
\NormalTok{pwp.fit.total}
\end{Highlighting}
\end{Shaded}

\begin{verbatim}
## Call:
## coxph(formula = Surv(start, stop, event) ~ as.factor(distrib) + 
##     strata(enum), data = subset(lymphoma_ag, enum < 4), cluster = id)
## 
##                       coef exp(coef) se(coef) robust se     z      p
## as.factor(distrib)1 0.5890    1.8023   0.5367    0.3874 1.521 0.1284
## as.factor(distrib)2 0.7086    2.0312   0.5505    0.4286 1.653 0.0982
## as.factor(distrib)3 0.5210    1.6838   0.7845    0.4616 1.129 0.2590
## 
## Likelihood ratio test=1.87  on 3 df, p=0.6009
## n= 101, number of events= 43 
##    (2 observations deleted due to missingness)
\end{verbatim}

Podem observar dels resultat, que cap dels coeficients del model és
significatiu i que la incorporació per estrats disminueix el risc en
totes tres categories respecte el model AG.

En canvi, encara que reduisca el risc, el que sí que es pot veure, es
que es manté la relació de que la lesió en més d'un lloc nodal durant el
diagnostic continua matinguent un risc superior que en els altres casos.

\subparagraph{3. Model PWP-Gap time}\label{model-pwp-gap-time}

El model PWP-Gap time és una extensió del model de Cox per a
esdeveniments recurrents que es centra en el temps entre events
consecutius (gap time). A diferència del model AG, que considera el
temps total desde el inici de seguiment, el model de PWP-Gap time modela
el temps entre la ocurrència d'un esdeveniment i el següent. Açò és útil
quan el risc d'un nou esdevenimient depen del temps transcurrit des de
l'últim esdeveniment.

El model PWP-Gap time assumeix que els temps entre esdeveniments
consecutius formen un procés de renovació. El risc per al k-èssim
esdeveniment del individu \texttt{i} es modela com:

\[
\lambda_k(t;Z_{ik}) = \lambda_{0k}(t) e^{\beta'Z_{it}(t)}
\] d'on:

\begin{itemize}
\tightlist
\item
  \(\lambda_{0k}(t)\) és la funció de risc basal per al k-èssim
  esdeveniment.
\item
  \(Z_{ik}(t)\) és un vector de covariables per al individu i en el
  k-èssim esdeveniment.
\item
  \(\beta\) és el vector de paràmetres a estimar.
\end{itemize}

En aquest model, cada esdeveniment pot tindre la seua propia funció de
risc basal (\(\lambda_{0k}(t)\)), lo que permiteix capturar diferències
en el risc entre esdeveniments successius. Endemés, dintre dels seus
suposits trobem:

\begin{itemize}
\item
  \emph{Procés de renovació}: els temps entre esdeveniments consecutius
  es suposen independents y la distribució del próxim temps entre events
  depen només del últim esdevenimment.
\item
  \emph{Risc basal especific del esdeveniment}: cada esdeveniment pot
  tindre una funció de risc basal diferent.
\item
  \emph{Efecte proporcional de les covariables}: les covariables tenen
  un efecte multiplicatiu constant sobre el risc, independent del número
  d'esdeveniments previs.
\end{itemize}

En el model PWP-Gap time, les dades tenen que estructurar-se en el
format de temps entre esdeveniments (gap time), és a dir, cada fila té
que representar el temps transcurrit des de l'últim esdeveniment fins al
següent. La següent funció ens ajudarà a fer-ho:

\begin{Shaded}
\begin{Highlighting}[]
\NormalTok{convert\_gap\_time }\OtherTok{\textless{}{-}} \ControlFlowTok{function}\NormalTok{(df, colID, colStop, colStart ) \{}
\NormalTok{  df\_gap }\OtherTok{\textless{}{-}}\NormalTok{ df}
\NormalTok{  df\_gap }\OtherTok{\textless{}{-}}\NormalTok{ df[}\FunctionTok{order}\NormalTok{(df[[colID]], df[[colStop]]), ]  }\CommentTok{\# ordenar per individu i temps}

  \CommentTok{\# Inicialitzar nova columna \textquotesingle{}gap\_time\textquotesingle{}}
\NormalTok{  df\_gap}\SpecialCharTok{$}\NormalTok{gap\_stop }\OtherTok{\textless{}{-}} \ConstantTok{NA}

\NormalTok{  ids }\OtherTok{\textless{}{-}} \FunctionTok{unique}\NormalTok{(df[[colID]])}
  \ControlFlowTok{for}\NormalTok{ (i }\ControlFlowTok{in}\NormalTok{ ids) \{}
\NormalTok{    idx }\OtherTok{\textless{}{-}} \FunctionTok{which}\NormalTok{(df[[colID]] }\SpecialCharTok{==}\NormalTok{ i)}
\NormalTok{    n }\OtherTok{\textless{}{-}} \FunctionTok{length}\NormalTok{(idx)}

    \CommentTok{\# Calcular temps entre esdeveniments consecutius}
\NormalTok{    times }\OtherTok{\textless{}{-}}\NormalTok{ df[[colStop]][idx]}
\NormalTok{    gaps }\OtherTok{\textless{}{-}} \FunctionTok{c}\NormalTok{(times[}\DecValTok{1}\NormalTok{], }\FunctionTok{diff}\NormalTok{(times))}
\NormalTok{    df\_gap}\SpecialCharTok{$}\NormalTok{gap\_stop[idx] }\OtherTok{\textless{}{-}}\NormalTok{ gaps}
\NormalTok{  \}}

\NormalTok{  df\_gap}\SpecialCharTok{$}\NormalTok{start }\OtherTok{\textless{}{-}} \DecValTok{0}
\NormalTok{  df\_gap}\SpecialCharTok{$}\NormalTok{stop }\OtherTok{\textless{}{-}}\NormalTok{ df\_gap}\SpecialCharTok{$}\NormalTok{gap\_stop}
\NormalTok{  df\_gap}\SpecialCharTok{$}\NormalTok{gap\_stop }\OtherTok{\textless{}{-}} \ConstantTok{NULL}  \CommentTok{\# opcional: eliminar columna auxiliar}

  \FunctionTok{return}\NormalTok{(df\_gap)}
\NormalTok{\}}
\end{Highlighting}
\end{Shaded}

Aplicant la funció a les nostres dades per obtindre el format correcte:

\begin{Shaded}
\begin{Highlighting}[]
\NormalTok{lymphoma\_pwp }\OtherTok{\textless{}{-}} \FunctionTok{convert\_gap\_time}\NormalTok{(lymphoma\_ag, }\AttributeTok{colID =} \StringTok{"id"}\NormalTok{, }\AttributeTok{colStop =} \StringTok{"stop"}\NormalTok{, }\AttributeTok{colStart =} \StringTok{"start"}\NormalTok{)}
\FunctionTok{head}\NormalTok{(lymphoma\_pwp)}
\end{Highlighting}
\end{Shaded}

\begin{verbatim}
##   id      time start      stop event enum delay age sex distrib effage
## 1  6  3.900826     0  3.900826     0    1    17  79   1       1     CR
## 2  7 63.173554     0 63.173554     0    1    33  25   1       1     CR
## 3  8 41.289256     0 41.289256     0    1    26  37   1       2     CR
## 6 11 17.950413     0 17.950413     0    1    31  43   2       2     CR
## 5 11 20.826446     0  2.876033     1    2    31  43   2       2     CR
## 4 11 29.421488     0  8.595041     1    3    31  43   2       2     CR
\end{verbatim}

Ara, realitzem el model PWP-Gap time amb les dades en el format adequat:

\begin{Shaded}
\begin{Highlighting}[]
\NormalTok{pwp.fit }\OtherTok{\textless{}{-}} \FunctionTok{coxph}\NormalTok{(}\FunctionTok{Surv}\NormalTok{(start, stop, event) }\SpecialCharTok{\textasciitilde{}} \FunctionTok{as.factor}\NormalTok{(distrib)}\SpecialCharTok{+}
                  \FunctionTok{cluster}\NormalTok{(id) }\SpecialCharTok{+} \FunctionTok{strata}\NormalTok{(enum), }\AttributeTok{data=}\NormalTok{lymphoma\_pwp)}
\NormalTok{pwp.fit}
\end{Highlighting}
\end{Shaded}

\begin{verbatim}
## Call:
## coxph(formula = Surv(start, stop, event) ~ as.factor(distrib) + 
##     strata(enum), data = lymphoma_pwp, cluster = id)
## 
##                       coef exp(coef) se(coef) robust se     z       p
## as.factor(distrib)1 0.7856    2.1937   0.5161    0.3363 2.336 0.01949
## as.factor(distrib)2 1.0536    2.8680   0.5235    0.3416 3.085 0.00204
## as.factor(distrib)3 0.5410    1.7177   0.7741    0.4099 1.320 0.18690
## 
## Likelihood ratio test=4.9  on 3 df, p=0.1789
## n= 110, number of events= 49 
##    (2 observations deleted due to missingness)
\end{verbatim}

I amb el model PWP-Gap time també podem obtindre coeficients de
regressió per estrat, de la següent manera:

\begin{Shaded}
\begin{Highlighting}[]
\NormalTok{pwp.fit.strata }\OtherTok{\textless{}{-}} \FunctionTok{coxph}\NormalTok{(}\FunctionTok{Surv}\NormalTok{(start, stop, event) }\SpecialCharTok{\textasciitilde{}} \FunctionTok{as.factor}\NormalTok{(distrib)}\SpecialCharTok{*} \FunctionTok{strata}\NormalTok{(enum) }\SpecialCharTok{+}
                  \FunctionTok{cluster}\NormalTok{(id), }\AttributeTok{data=}\NormalTok{lymphoma\_pwp)}
\NormalTok{pwp.fit.strata}
\end{Highlighting}
\end{Shaded}

\begin{verbatim}
## Call:
## coxph(formula = Surv(start, stop, event) ~ as.factor(distrib) + 
##     strata(enum) + as.factor(distrib):strata(enum), data = lymphoma_pwp, 
##     cluster = id)
## 
##                                              coef  exp(coef)   se(coef)
## as.factor(distrib)1                     1.074e+00  2.927e+00  8.168e-01
## as.factor(distrib)2                     1.365e+00  3.916e+00  8.666e-01
## as.factor(distrib)3                     1.616e+00  5.034e+00  1.233e+00
## as.factor(distrib)1:strata(enum)enum=2 -5.036e-01  6.043e-01  1.070e+00
## as.factor(distrib)2:strata(enum)enum=2 -7.492e-01  4.728e-01  1.111e+00
## as.factor(distrib)3:strata(enum)enum=2 -5.350e-01  5.857e-01  1.706e+00
## as.factor(distrib)1:strata(enum)enum=3  8.891e-01  2.433e+00  1.577e+00
## as.factor(distrib)2:strata(enum)enum=3  9.003e-01  2.460e+00  1.579e+00
## as.factor(distrib)3:strata(enum)enum=3         NA         NA  0.000e+00
## as.factor(distrib)1:strata(enum)enum=4  1.784e+01  5.605e+07  5.003e+03
## as.factor(distrib)2:strata(enum)enum=4  1.945e+01  2.809e+08  5.003e+03
## as.factor(distrib)3:strata(enum)enum=4         NA         NA  0.000e+00
##                                         robust se      z      p
## as.factor(distrib)1                     7.872e-01  1.364 0.1725
## as.factor(distrib)2                     8.264e-01  1.652 0.0986
## as.factor(distrib)3                     1.283e+00  1.260 0.2078
## as.factor(distrib)1:strata(enum)enum=2  1.189e+00 -0.424 0.6718
## as.factor(distrib)2:strata(enum)enum=2  1.273e+00 -0.588 0.5562
## as.factor(distrib)3:strata(enum)enum=2  1.539e+00 -0.348 0.7282
## as.factor(distrib)1:strata(enum)enum=3  1.408e+00  0.631 0.5278
## as.factor(distrib)2:strata(enum)enum=3  1.390e+00  0.648 0.5172
## as.factor(distrib)3:strata(enum)enum=3  0.000e+00     NA     NA
## as.factor(distrib)1:strata(enum)enum=4  2.080e+00  8.579 <2e-16
## as.factor(distrib)2:strata(enum)enum=4  2.211e+00  8.798 <2e-16
## as.factor(distrib)3:strata(enum)enum=4  0.000e+00     NA     NA
## 
## Likelihood ratio test=10.5  on 10 df, p=0.3981
## n= 110, number of events= 49 
##    (2 observations deleted due to missingness)
\end{verbatim}

Cal tenir en compte que els coeficients d'algunes covariables no es
poden estimar. Si el nombre de subjectes disminueix a mesura que
augmenta \(k\) (esdeveniment), no es poden obtenir estimacions de
coeficients estables per a aquests rangs més alts, \(k\), de recaigudes.
Veiem que, els únics valors on el model no té la capacitat d'obtindre
els NA son en la categoria de lessió generalitzada quan el nombre de
racaigudes és superior a 2. Per solventar aquesta problema podem limitar
el nombre de recaigudes a 3 i que el model sí tinga la capacitat de
ajustar correctament els coeficients, com es mostra:

\begin{Shaded}
\begin{Highlighting}[]
\NormalTok{pwp.fit.strata}\FloatTok{.3} \OtherTok{\textless{}{-}} \FunctionTok{coxph}\NormalTok{(}\FunctionTok{Surv}\NormalTok{(start, stop, event) }\SpecialCharTok{\textasciitilde{}} \FunctionTok{as.factor}\NormalTok{(distrib) }\SpecialCharTok{+} \FunctionTok{strata}\NormalTok{(enum) }\SpecialCharTok{+}
                  \FunctionTok{cluster}\NormalTok{(id) , }\AttributeTok{data=}\FunctionTok{subset}\NormalTok{(lymphoma\_pwp, enum}\SpecialCharTok{\textless{}}\DecValTok{3}\NormalTok{))}
\NormalTok{pwp.fit.strata}\FloatTok{.3}
\end{Highlighting}
\end{Shaded}

\begin{verbatim}
## Call:
## coxph(formula = Surv(start, stop, event) ~ as.factor(distrib) + 
##     strata(enum), data = subset(lymphoma_pwp, enum < 3), cluster = id)
## 
##                       coef exp(coef) se(coef) robust se     z       p
## as.factor(distrib)1 0.8033    2.2330   0.5273    0.3754 2.140 0.03235
## as.factor(distrib)2 0.9333    2.5430   0.5477    0.3492 2.673 0.00752
## as.factor(distrib)3 1.3328    3.7916   0.8482    0.6001 2.221 0.02636
## 
## Likelihood ratio test=4.36  on 3 df, p=0.2255
## n= 87, number of events= 32 
##    (2 observations deleted due to missingness)
\end{verbatim}

Les funcions de supervivència de referència es podem estimar i
representar gràficament:

\begin{Shaded}
\begin{Highlighting}[]
\NormalTok{pwp.fit.strata}\FloatTok{.3} \OtherTok{\textless{}{-}} \FunctionTok{coxph}\NormalTok{(}\FunctionTok{Surv}\NormalTok{(start, stop, event) }\SpecialCharTok{\textasciitilde{}} \FunctionTok{as.factor}\NormalTok{(distrib) }\SpecialCharTok{+} \FunctionTok{strata}\NormalTok{(enum) }\SpecialCharTok{+} \FunctionTok{cluster}\NormalTok{(id),}\AttributeTok{data=}\FunctionTok{subset}\NormalTok{(lymphoma\_pwp, enum}\SpecialCharTok{\textless{}}\DecValTok{4}\NormalTok{))}

\FunctionTok{plot}\NormalTok{(}\FunctionTok{survfit}\NormalTok{(pwp.fit.strata}\FloatTok{.3}\NormalTok{),}\AttributeTok{lty=}\FunctionTok{c}\NormalTok{(}\DecValTok{1}\NormalTok{,}\DecValTok{2}\NormalTok{,}\DecValTok{3}\NormalTok{))}
\FunctionTok{legend}\NormalTok{(}\StringTok{"topright"}\NormalTok{, }\FunctionTok{c}\NormalTok{(}\StringTok{"1st recaiguda"}\NormalTok{, }\StringTok{"2nd recaiguda"}\NormalTok{, }\StringTok{"3rd recaiguda"}\NormalTok{), }\AttributeTok{lty=}\FunctionTok{c}\NormalTok{(}\DecValTok{1}\NormalTok{,}\DecValTok{2}\NormalTok{,}\DecValTok{3}\NormalTok{), }\AttributeTok{cex =} \FloatTok{0.7}\NormalTok{)}
\end{Highlighting}
\end{Shaded}

\includegraphics{events_recu_partII_files/figure-latex/unnamed-chunk-18-1.pdf}
gráficament, es pot observar les mateixes conclusions que hem deduït en
el model AG i la necessitat d'estratificar pel nombre de recaigudes
previes. Ací, s'observa que les estimacions de les funcions de
supervivència disminueixen significativament després de la primera
recaiguda al llarg del temps. Veiem que aquells individus que presenten
més d'una recaiguda la seua supervivència es prou inferior a aquells que
només experimenten una.

Podem veure com afecta al nombre de lessions del model AG:

\begin{Shaded}
\begin{Highlighting}[]
\NormalTok{ag.fit }\OtherTok{\textless{}{-}} \FunctionTok{coxph}\NormalTok{(}\FunctionTok{Surv}\NormalTok{(start, stop, event) }\SpecialCharTok{\textasciitilde{}} \FunctionTok{as.factor}\NormalTok{(distrib) }\SpecialCharTok{+}
                      \FunctionTok{cluster}\NormalTok{(id), }\AttributeTok{data=}\FunctionTok{subset}\NormalTok{(lymphoma\_ag, enum}\SpecialCharTok{\textless{}}\DecValTok{4}\NormalTok{))}

\FunctionTok{plot}\NormalTok{(}\FunctionTok{survfit}\NormalTok{(ag.fit.dep), }\AttributeTok{lty=}\FunctionTok{c}\NormalTok{(}\DecValTok{1}\NormalTok{,}\DecValTok{2}\NormalTok{,}\DecValTok{4}\NormalTok{)) }
\FunctionTok{legend}\NormalTok{(}\StringTok{"bottomleft"}\NormalTok{, }\FunctionTok{c}\NormalTok{(}\StringTok{"localitzada"}\NormalTok{, }\StringTok{"\textgreater{} 1 lloc nodal"}\NormalTok{, }\StringTok{"generalitzada"}\NormalTok{), }\AttributeTok{lty=}\FunctionTok{c}\NormalTok{(}\DecValTok{1}\NormalTok{,}\DecValTok{2}\NormalTok{,}\DecValTok{4}\NormalTok{), }\AttributeTok{cex =} \FloatTok{0.7}\NormalTok{)}
\end{Highlighting}
\end{Shaded}

\includegraphics{events_recu_partII_files/figure-latex/unnamed-chunk-19-1.pdf}

Veiem que, quan els individus presenten lessions en més d'un lloc nodal,
la seua funció de supervivència decau més rapidament que en els altres
condicions. Endemés, si presenta una lessió generalitzada, la seua
funció de supervivència és la més elevada, inclús al llarg del temps
experimenta un petit increment. Encara que en es altre casos açò no
passa.

\paragraph{3. Model WLW (Wei, Lin y
Weisfeld)}\label{model-wlw-wei-lin-y-weisfeld}

El model WLW és una extensió del model de COx per a esdeveniments
recurrents que permiteix utilitzar dades amb múltiples temps de fallada
incomplets. A diferència dels models AG i PWP, que es centren en el
procés de recompte o el temps entre esdeveniments, de model WLW modela
la distribució marginal de cada esdeveniment recurrent, permitint que
cada esdeveniment tinga la seua propia funció de risc basal.

El model assumeix que la funció de risc per a k-èssim esdeveniment del
individu i és:

\[
\lambda_k(t;Z_{ik}) = \lambda_{0k}(t) e^{\beta'Z_{it}(t)}
\] d'on:

\begin{itemize}
\tightlist
\item
  \(\lambda_{0k}(t)\) és la funció de risc basal per al k-èssim
  esdeveniment.
\item
  \(Z_{ik}(t)\) és un vector de covariables per al individu i en el
  k-èssim esdeveniment.
\item
  \(\beta\) és el vector de paràmetres a estimar.
\end{itemize}

En aquest model, cada esdeveniment pot tindre la seua propia funció de
risc basal (\(\lambda_{0k}(t)\)), lo que permiteix capturar diferències
en el risc entre esdeveniments successius. Endemés, dintre dels seus
suposits trobem:

\begin{itemize}
\item
  \emph{Indepència condicional}: els temps entre esdeveniments
  recurrents son independents condicionalment a les covariables
\item
  \emph{Risc basal especific del esdeveniment}: cada esdeveniment pot
  tindre una funció de risc basal diferent.
\item
  \emph{Efecte proporcional de les covariables}: les covariables tenen
  un efecte multiplicatiu constant sobre el risc, independent del número
  d'esdeveniments previs.
\end{itemize}

En el model WLW, les dades tenen que estructurar-se en el format de
temps total des de l'inici, és a dir, cada fila té que representar el
temps transcurrit des de l'últim esdeveniment fins al següent i que cada
pacient tinga el mateix nombre d'entrades. Per example podem fixar el
nombre de recaigudes a 3, però els casos que presenten menys, tenim que
duplicar-les. La següent funció ens dona el data ven estructurat

\begin{Shaded}
\begin{Highlighting}[]
\NormalTok{conver\_total\_time }\OtherTok{\textless{}{-}} \ControlFlowTok{function}\NormalTok{(df, colID, }\AttributeTok{lim=}\DecValTok{3}\NormalTok{)\{}
\NormalTok{  df.total }\OtherTok{\textless{}{-}} \ConstantTok{NULL}
\NormalTok{  ids }\OtherTok{\textless{}{-}} \FunctionTok{unique}\NormalTok{(df[[colID]])}
  \ControlFlowTok{for}\NormalTok{ (i }\ControlFlowTok{in}\NormalTok{ ids)\{}
\NormalTok{    aux }\OtherTok{\textless{}{-}} \FunctionTok{sum}\NormalTok{(df[colID]}\SpecialCharTok{==}\NormalTok{i)}
\NormalTok{    mat }\OtherTok{\textless{}{-}}\NormalTok{ df[df[,colID]}\SpecialCharTok{==}\NormalTok{i, ]}
    
    \ControlFlowTok{if}\NormalTok{(aux }\SpecialCharTok{\textless{}}\NormalTok{ lim)\{}
\NormalTok{      df.total }\OtherTok{\textless{}{-}} \FunctionTok{rbind}\NormalTok{(df.total, mat)}
\NormalTok{      ultima }\OtherTok{\textless{}{-}}\NormalTok{ mat[aux,]}
      \ControlFlowTok{while}\NormalTok{(aux }\SpecialCharTok{\textless{}}\NormalTok{ lim)\{}
\NormalTok{        df.total }\OtherTok{\textless{}{-}} \FunctionTok{rbind}\NormalTok{(df.total, ultima)}
\NormalTok{        aux }\OtherTok{\textless{}{-}}\NormalTok{ aux }\SpecialCharTok{+}\DecValTok{1}
\NormalTok{      \}}
\NormalTok{    \} }\ControlFlowTok{else}\NormalTok{\{}
\NormalTok{      df.total }\OtherTok{\textless{}{-}} \FunctionTok{rbind}\NormalTok{(df.total, mat)}
\NormalTok{    \}}
\NormalTok{  \}}
  
  \FunctionTok{return}\NormalTok{(df.total)}
\NormalTok{\}}
\end{Highlighting}
\end{Shaded}

Aplicant aquesta funció a la nostra data set anterior:

\begin{Shaded}
\begin{Highlighting}[]
\NormalTok{lymphoma\_wlw }\OtherTok{\textless{}{-}} \FunctionTok{conver\_total\_time}\NormalTok{(lymphoma\_ag, }\AttributeTok{colID =} \StringTok{"id"}\NormalTok{)}
\FunctionTok{head}\NormalTok{(lymphoma\_wlw)}
\end{Highlighting}
\end{Shaded}

\begin{verbatim}
##    id      time start      stop event enum delay age sex distrib effage
## 1   6  3.900826     0  3.900826     0    1    17  79   1       1     CR
## 11  6  3.900826     0  3.900826     0    1    17  79   1       1     CR
## 12  6  3.900826     0  3.900826     0    1    17  79   1       1     CR
## 2   7 63.173554     0 63.173554     0    1    33  25   1       1     CR
## 21  7 63.173554     0 63.173554     0    1    33  25   1       1     CR
## 22  7 63.173554     0 63.173554     0    1    33  25   1       1     CR
\end{verbatim}

Un cop tenim les dades en el format adequat, li passem les dades al
model:

\begin{Shaded}
\begin{Highlighting}[]
\NormalTok{wlw.fit.total }\OtherTok{\textless{}{-}} \FunctionTok{coxph}\NormalTok{(}\FunctionTok{Surv}\NormalTok{(time, event) }\SpecialCharTok{\textasciitilde{}} \FunctionTok{as.factor}\NormalTok{(distrib) }\SpecialCharTok{+} \FunctionTok{strata}\NormalTok{(enum) }\SpecialCharTok{+}
                  \FunctionTok{cluster}\NormalTok{(id) , }\AttributeTok{data=}\FunctionTok{subset}\NormalTok{(lymphoma\_wlw, enum}\SpecialCharTok{\textless{}}\DecValTok{4}\NormalTok{))}
\NormalTok{wlw.fit.total}
\end{Highlighting}
\end{Shaded}

\begin{verbatim}
## Call:
## coxph(formula = Surv(time, event) ~ as.factor(distrib) + strata(enum), 
##     data = subset(lymphoma_wlw, enum < 4), cluster = id)
## 
##                       coef exp(coef) se(coef) robust se     z      p
## as.factor(distrib)1 0.5564    1.7443   0.4378    0.4332 1.284 0.1991
## as.factor(distrib)2 0.7496    2.1162   0.4366    0.4616 1.624 0.1043
## as.factor(distrib)3 0.9701    2.6382   0.7023    0.5445 1.782 0.0748
## 
## Likelihood ratio test=3.68  on 3 df, p=0.2976
## n= 189, number of events= 51
\end{verbatim}

\begin{Shaded}
\begin{Highlighting}[]
\FunctionTok{plot}\NormalTok{(}\FunctionTok{survfit}\NormalTok{(wlw.fit.total), , }\AttributeTok{lty=}\FunctionTok{c}\NormalTok{(}\DecValTok{1}\NormalTok{,}\DecValTok{2}\NormalTok{,}\DecValTok{3}\NormalTok{)) }
\FunctionTok{legend}\NormalTok{(}\StringTok{"topright"}\NormalTok{, }\FunctionTok{c}\NormalTok{(}\StringTok{"1st recaiguda"}\NormalTok{, }\StringTok{"2nd recaiguda"}\NormalTok{, }\StringTok{"3rd recaiguda"}\NormalTok{), }\AttributeTok{lty=}\FunctionTok{c}\NormalTok{(}\DecValTok{1}\NormalTok{,}\DecValTok{2}\NormalTok{,}\DecValTok{3}\NormalTok{), }\AttributeTok{cex =} \FloatTok{0.7}\NormalTok{)}
\end{Highlighting}
\end{Shaded}

\includegraphics{events_recu_partII_files/figure-latex/unnamed-chunk-23-1.pdf}

\begin{itemize}
\tightlist
\item
  \emph{¿Obtenemos la misma conclusión utilizando los tres modelos?
  (NOTA: utilice algunas de las funciones que hemos visto en las clases
  para preparar los datos necesarios).}
\end{itemize}

Per veure si els tres modeles presenten resultat semblant, que com podem
esperar, el model AG serà diferent dels altres dos ja que aquests últims
consideren que els incrementes no son independents. Aleshora, tenim que
observar els coeficients dels models:

\begin{verbatim}
##      distrib1  distrib2  distrib3
## AG  0.9471567 1.3788592 1.4745074
## PWP 0.7855832 1.0536113 0.5409816
## WLW 0.5563699 0.7496388 0.9700970
\end{verbatim}

\begin{itemize}
\tightlist
\item
  \emph{Repita los mismos análisis ajustando por sexo y respuesta al
  tratamiento (variable tt effage). ¿Obtenemos la misma conclusión que
  en los modelos sin dicho ajuste?}
\end{itemize}

Ajustem el model AG per sexe i resposta al tractament

\begin{Shaded}
\begin{Highlighting}[]
\NormalTok{ag.fit.adj }\OtherTok{\textless{}{-}} \FunctionTok{coxph}\NormalTok{(}\FunctionTok{Surv}\NormalTok{(start, stop, event) }\SpecialCharTok{\textasciitilde{}} \FunctionTok{as.factor}\NormalTok{(distrib)}\SpecialCharTok{+}
                  \FunctionTok{cluster}\NormalTok{(id) }\SpecialCharTok{+} \FunctionTok{as.factor}\NormalTok{(sex) }\SpecialCharTok{+} \FunctionTok{as.factor}\NormalTok{(effage), }\AttributeTok{data=}\NormalTok{lymphoma\_ag)}
\end{Highlighting}
\end{Shaded}

\begin{verbatim}
## Warning in Surv(start, stop, event): Stop time must be > start time, NA created
\end{verbatim}

\begin{Shaded}
\begin{Highlighting}[]
\NormalTok{ag.fit.adj}
\end{Highlighting}
\end{Shaded}

\begin{verbatim}
## Call:
## coxph(formula = Surv(start, stop, event) ~ as.factor(distrib) + 
##     as.factor(sex) + as.factor(effage), data = lymphoma_ag, cluster = id)
## 
##                       coef exp(coef) se(coef) robust se     z        p
## as.factor(distrib)1 1.5476    4.7002   0.5615    0.5577 2.775 0.005521
## as.factor(distrib)2 1.9322    6.9049   0.5684    0.5412 3.570 0.000356
## as.factor(distrib)3 1.7824    5.9443   0.8121    0.7884 2.261 0.023768
## as.factor(sex)2     0.6861    1.9859   0.3391    0.3489 1.966 0.049253
## as.factor(effage)PR 0.1912    1.2107   0.5122    0.5765 0.332 0.740147
## as.factor(effage)SD 1.6751    5.3395   1.0906    1.2208 1.372 0.170029
## 
## Likelihood ratio test=17.03  on 6 df, p=0.009187
## n= 110, number of events= 49 
##    (2 observations deleted due to missingness)
\end{verbatim}

\begin{Shaded}
\begin{Highlighting}[]
\NormalTok{pwp.fit.total.adj }\OtherTok{\textless{}{-}} \FunctionTok{coxph}\NormalTok{(}\FunctionTok{Surv}\NormalTok{(start, stop, event) }\SpecialCharTok{\textasciitilde{}} \FunctionTok{as.factor}\NormalTok{(distrib) }\SpecialCharTok{+} \FunctionTok{cluster}\NormalTok{(id) }\SpecialCharTok{+} \FunctionTok{strata}\NormalTok{(enum) }\SpecialCharTok{+}\FunctionTok{as.factor}\NormalTok{(sex) }\SpecialCharTok{+} \FunctionTok{as.factor}\NormalTok{(effage), }\AttributeTok{data=}\FunctionTok{subset}\NormalTok{(lymphoma\_ag, enum }\SpecialCharTok{\textless{}} \DecValTok{4}\NormalTok{))}
\end{Highlighting}
\end{Shaded}

\begin{verbatim}
## Warning in Surv(start, stop, event): Stop time must be > start time, NA created
\end{verbatim}

\begin{Shaded}
\begin{Highlighting}[]
\NormalTok{pwp.fit.total.adj}
\end{Highlighting}
\end{Shaded}

\begin{verbatim}
## Call:
## coxph(formula = Surv(start, stop, event) ~ as.factor(distrib) + 
##     strata(enum) + as.factor(sex) + as.factor(effage), data = subset(lymphoma_ag, 
##     enum < 4), cluster = id)
## 
##                           coef  exp(coef)   se(coef)  robust se     z      p
## as.factor(distrib)1  1.0962060  2.9927897  0.6146930  0.4921437 2.227 0.0259
## as.factor(distrib)2  1.2301217  3.4216458  0.6299074  0.4870274 2.526 0.0115
## as.factor(distrib)3  1.1911575  3.2908881  0.8730001  0.5998920 1.986 0.0471
## as.factor(sex)2      0.6219273  1.8625142  0.3895888  0.2814146 2.210 0.0271
## as.factor(effage)PR -0.0001036  0.9998964  0.5949579  0.5701175 0.000 0.9999
## as.factor(effage)SD  1.9308135  6.8951170  1.1228109  1.1719176 1.648 0.0994
## 
## Likelihood ratio test=6.37  on 6 df, p=0.3832
## n= 101, number of events= 43 
##    (2 observations deleted due to missingness)
\end{verbatim}

Ara fem el mateix per al model PWP-Gap time:

\begin{Shaded}
\begin{Highlighting}[]
\NormalTok{pwp.fit.adj }\OtherTok{\textless{}{-}} \FunctionTok{coxph}\NormalTok{(}\FunctionTok{Surv}\NormalTok{(start, stop, event) }\SpecialCharTok{\textasciitilde{}} \FunctionTok{as.factor}\NormalTok{(distrib)}\SpecialCharTok{+}
                  \FunctionTok{cluster}\NormalTok{(id) }\SpecialCharTok{+} \FunctionTok{strata}\NormalTok{(enum) }\SpecialCharTok{+} \FunctionTok{as.factor}\NormalTok{(sex) }\SpecialCharTok{+} \FunctionTok{as.factor}\NormalTok{(effage), }\AttributeTok{data=}\FunctionTok{subset}\NormalTok{(lymphoma\_pwp,enum}\SpecialCharTok{\textless{}}\DecValTok{4}\NormalTok{))}
\end{Highlighting}
\end{Shaded}

\begin{verbatim}
## Warning in Surv(start, stop, event): Stop time must be > start time, NA created
\end{verbatim}

\begin{Shaded}
\begin{Highlighting}[]
\NormalTok{pwp.fit.adj}
\end{Highlighting}
\end{Shaded}

\begin{verbatim}
## Call:
## coxph(formula = Surv(start, stop, event) ~ as.factor(distrib) + 
##     strata(enum) + as.factor(sex) + as.factor(effage), data = subset(lymphoma_pwp, 
##     enum < 4), cluster = id)
## 
##                       coef exp(coef) se(coef) robust se     z        p
## as.factor(distrib)1 1.2229    3.3972   0.5886    0.4033 3.032 0.002426
## as.factor(distrib)2 1.3379    3.8110   0.5991    0.3988 3.355 0.000794
## as.factor(distrib)3 1.5677    4.7958   0.8543    0.5732 2.735 0.006233
## as.factor(sex)2     0.4673    1.5957   0.3699    0.2520 1.854 0.063695
## as.factor(effage)PR 0.4835    1.6217   0.5552    0.5891 0.821 0.411857
## as.factor(effage)SD 2.1222    8.3491   1.1235    1.0903 1.946 0.051598
## 
## Likelihood ratio test=8.54  on 6 df, p=0.2012
## n= 101, number of events= 43 
##    (2 observations deleted due to missingness)
\end{verbatim}

i pel model WLW:

\begin{Shaded}
\begin{Highlighting}[]
\NormalTok{wlw.fit.total.adj }\OtherTok{\textless{}{-}} \FunctionTok{coxph}\NormalTok{(}\FunctionTok{Surv}\NormalTok{(time, event) }\SpecialCharTok{\textasciitilde{}} \FunctionTok{as.factor}\NormalTok{(distrib) }\SpecialCharTok{+} \FunctionTok{strata}\NormalTok{(enum) }\SpecialCharTok{+}
                  \FunctionTok{cluster}\NormalTok{(id) }\SpecialCharTok{+} \FunctionTok{as.factor}\NormalTok{(sex) }\SpecialCharTok{+} \FunctionTok{as.factor}\NormalTok{(effage), }\AttributeTok{data=}\FunctionTok{subset}\NormalTok{(lymphoma\_wlw, enum}\SpecialCharTok{\textless{}}\DecValTok{4}\NormalTok{))}
\NormalTok{wlw.fit.total.adj}
\end{Highlighting}
\end{Shaded}

\begin{verbatim}
## Call:
## coxph(formula = Surv(time, event) ~ as.factor(distrib) + strata(enum) + 
##     as.factor(sex) + as.factor(effage), data = subset(lymphoma_wlw, 
##     enum < 4), cluster = id)
## 
##                        coef exp(coef) se(coef) robust se     z      p
## as.factor(distrib)1  1.1442    3.1398   0.5330    0.6565 1.743 0.0814
## as.factor(distrib)2  1.2130    3.3636   0.5072    0.5915 2.051 0.0403
## as.factor(distrib)3  1.7193    5.5807   0.7849    0.8253 2.083 0.0372
## as.factor(sex)2      0.6673    1.9489   0.3723    0.4667 1.430 0.1528
## as.factor(effage)PR  0.6060    1.8331   0.5437    0.5242 1.156 0.2477
## as.factor(effage)SD  2.7671   15.9119   1.1039    1.1897 2.326 0.0200
## 
## Likelihood ratio test=11.12  on 6 df, p=0.08479
## n= 189, number of events= 51
\end{verbatim}

Ara de la mateixa forma que abans recollim tota la informació en la
següent taula:

\begin{verbatim}
##           distrib1 distrib2 distrib3      sex2      effagePR effageSD
## AG        1.547615 1.932237 1.782433 0.6860770  0.1911997164 1.675123
## PWP-Total 1.096206 1.230122 1.191157 0.6219273 -0.0001036279 1.930813
## PWP-Gap   1.222941 1.337894 1.567747 0.4673196  0.4834619590 2.122156
## WLW       1.144171 1.213006 1.719320 0.6672706  0.6060317095 2.767069
\end{verbatim}

Si ajustem per sexe i per el diagnostic, el resultats no son igual que
abans, però les relacions sí, en el model AG adjustat es continua
mantenint la relacions entre les categories de lessions durant el
diagnostic ja que la categoria que presenta més risc respecte les altres
és la associada a lesió en més d'un lloc nodal respecte les altres. En
canvi, en els altres model, el comportament de que la lesió
generalitzada és la que presenta major risc. Per que respecta a les
variables ajustades, el sex 2 té un major risc respecte el sexe 1 i el
mateix passa pels diagnostics PR i SD respecte CD.

Endemés, en el model PWP-Total time el risc de tindre un diagnostic
``PR'' disminueix el risc de patir recaiguda respecte la categoria
basal.

\paragraph{Conclusió}\label{conclusiuxf3}

\emph{Sense ajust per covariables}

Els resultats dels models AG, PWP-Gap time, PWP-Total time i WLW mostren
diferències en els coeficients de la variable \texttt{distrib}. Això
indica que el nombre de lesions en el diagnòstic té un efecte
significatiu en el temps fins a la recaiguda, però la magnitud d'aquest
efecte varia segons el model utilitzat.

\emph{Amb ajust per covariables (sexe i resposta al tractament)}

Després d'ajustar per sexe i resposta al tractament, els coeficients de
la variable \texttt{distrib} continuen sent significatius, però els seus
valors canvien lleugerament. Això suggereix que les covariables
addicionals expliquen part de la variabilitat en el temps fins a la
recaiguda, però el nombre de lesions continua sent un factor important.

\emph{Comparació de models}

\begin{itemize}
\item
  Els models AG i PWP-Total time produeixen resultats similars, cosa que
  indica que l'efecte del nombre de lesions és consistent quan es
  considera el temps total des de l'inici del seguiment.
\item
  El model PWP-Gap time mostra diferències més marcades, la qual cosa
  podria reflectir que el temps entre recaigudes consecutives està
  influït per altres factors no capturats en el model.
\item
  El model WLW també mostra resultats similars als models AG i PWP-Total
  time, cosa que suggereix que l'efecte del nombre de lesions és robust
  a diferents formulacions temporals.
\end{itemize}

En conclusió, tots els models coincideixen que el nombre de lesions en
el diagnòstic és un predictor significatiu del temps fins a la recaiguda
en pacients amb limfoma de grau baix. Tanmateix, la magnitud exacta
d'aquest efecte pot variar segons el model i les covariables incloses.

\end{document}
